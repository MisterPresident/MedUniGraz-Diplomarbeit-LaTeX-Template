% Main Thesis File: This is the main file and it connects all the different parts of the thesis and compiles it into a single outcome file.

\documentclass[onecolumn, 12pt, a4paper]{article}
\renewcommand{\baselinestretch}{1.5} % baseline stretch

% Packages
\usepackage{array}
\usepackage{amssymb}
\usepackage{amsthm}
\usepackage[cmex10]{amsmath}
\usepackage{adjustbox} 
\usepackage{arabtex} 
\usepackage{academicons}
\usepackage[german]{babel}
\usepackage{breakcites} 
\usepackage{color}
\usepackage{epsfig}
\usepackage{etoolbox}
\usepackage{enumitem}
\usepackage{float}
\usepackage{fancyhdr}
\usepackage{fontspec}
 

\usepackage{graphicx}
\usepackage{graphics}
%\usepackage{hyperref} % if you want to highlight the links
\usepackage[hidelinks]{hyperref}
\usepackage{latexsym,amsfonts}
\usepackage{longtable}
\usepackage{listings}
\usepackage{lscape} 
\usepackage{lipsum}
\usepackage{multirow}             
\usepackage{pdfpages}
\usepackage[figuresright]{rotating} 
\usepackage{sectsty}
\usepackage{setspace}
\usepackage{subfigure}
\usepackage{textcomp}
\usepackage[subfigure]{tocloft}
\usepackage{utf8} 
\usepackage{url} 
\usepackage{wrapfig} 
\usepackage{wasysym} 
\usepackage{xcolor}

% parameters
\def\thesistype {Diplomarbeit}
\title{Titel}
\def\subtitle {Untertitel (optional)}
\author{Vorname Zuname}
\date{01.01.0001}
\def\place {Graz}
\def\degree {Doktor(in) der gesamten Heilkunde\\(Dr. med. univ.) }
\def\institut {Institut / Klinik für ...}
\def\supervisor {Supervisor}
\def\uni {Medizinische Universität Graz}

% here you can define your own acronymes

% Changing chapters' headings and subheadings to size 14
\setmainfont{Arial}
\sectionfont{\fontsize{16}{18}\selectfont}
\subsectionfont{\fontsize{14}{15}\selectfont}
\subsubsectionfont{\fontsize{14}{15}\selectfont}

       
\pagestyle{fancy} % adds a line at the top of every page, except title-page
\renewcommand{\headrulewidth}{0pt}
\fancyhf{} % Start with clearing everything in the header and footer
\fancyfoot[R]{\thepage}


%The following code changes the empty vertical space above a new chapter title. It sets it from 50pt to 20pt
\makeatletter
\patchcmd{\@makechapterhead}{50\p@}{20pt}{}{}
\patchcmd{\@makeschapterhead}{50\p@}{20pt}{}{}
\makeatother
%end of modification



% The following code redefines the plain pagestyle with the objective of moving the page number from the bottom to the top of the page. This only affects new chapter pages.
\fancypagestyle{plain}{
\fancyhf{} %clear all header and footer fields
\fancyfoot[R]{\thepage} %puts number on top center of the page
\renewcommand{\headrulewidth}{0pt}
\renewcommand{\footrulewidth}{0pt}}
%ending of plain pagestyle modification.




% ### Nomenclature, List of Abbreviations and List of Symbols 
   \usepackage{ifthen,xkeyval,xfor,amsgen}
   \usepackage[acronym,toc, nogroupskip]{glossaries}
   \newglossary[slg]{symbols}{syi}{sbl}{List of Symbols}
 
   \makeglossaries
   
   \newglossaryentry{latex}
{
        name=latex,
        description={Is a mark up language specially suited for 
scientific documents}
}

\newglossaryentry{maths}
{
        name=mathematics,
        description={Mathematics is what mathematicians do}
}

\newglossaryentry{formula}
{
        name=formula,
        description={A mathematical expression}
}

\newacronym{gcd}{GCD}{Greatest Common Divisor}

\newacronym{lcm}{LCM}{Least Common Multiple}

\newacronym{mug}{MUG}{Medizinische Universität Graz}

   % Run the following three lines in the command line to get the lists
%makeindex -s Thesis.ist -t Thesis.alg -o Thesis.acr Thesis.acn
%makeindex -s Thesis.ist -t Thesis.slg -o Thesis.syi Thesis.sbl
%makeindex -s Thesis.ist -t Thesis.glg -o Thesis.gls Thesis.glo

% ### End of addition




% Modified commands
\newcommand{\Tab}{\hspace{2ex}}
\usepackage[lmargin=30mm, rmargin=25mm, vmargin=25mm, headsep=2.5mm]{geometry}
\newcommand{\mathsym}[1]{{}}
\newcommand{\unicode}[1]{{}}
%\newcommand{\orcid}[1]{\href{https://orcid.org/#1}{\textcolor[HTML]{A6CE39}{\aiOrcid}}}
\newcommand{\orcid}{\includegraphics[width=8pt]{ORCID}} % ORCID ID


\begin{document}
% $$$$$$$$$$$$$$$$$  Start of Thesis Front matter   $$$$$$$$$$$$$$

%\vspace{10mm}  % vertical space
\thispagestyle{empty}
\addvspace{5mm}  % vertical space until length

%$$$$$$$$$$$$$$$$$$$$$$$$$$$$$$$$$$$$$$$$$$$$$$$$$$$$$$$$$$$$$$$$$$$$$$$$$$$$$$$$$

% make the title page
\begin{titlepage}
    \makeatletter
    \thispagestyle{empty}
    \doublespacing
    \begin{center}
        {\textbf\thesistype} \\
        \vspace{1cm}
        {\LARGE\textbf\textsc{\@title}} \\
        {\Large\textbf\subtitle} \\
        \vspace{2cm}
        eingereicht von\\
        {\large\textbf \@author} \\
        \vspace{2cm}
        zur Erlangung des akademischen Grades \\
        \vspace{0.2cm}
        {\large\textbf\degree} \\
        an der \\
        \vspace{2cm}
        {\large\textbf{\uni}} \\
        \vspace{0.3cm}
        ausgeführt am \\
        {\large\textbf\institut} \\
        \vspace{0.3cm}
        unter der Anleitung von Betreuer/innen \\
        {\large\textbf\supervisor{}} \\
    \end{center}
    \vfill
    \place, \@date
    \makeatother
\end{titlepage}

\section*{Eidesstattliche Erklärung}

\begin{itshape}
Ich erkläre ehrenwörtlich, dass ich die vorliegende Arbeit selbstständig und ohne
fremde Hilfe verfasst habe, andere als die angegebenen Quellen nicht verwendet
habe und die den benutzten Quellen wörtlich oder inhaltlich entnommenen Stellen
als solche kenntlich gemacht habe.

\vspace{2cm}
\makeatletter
\noindent \place, \@date \hfill \@author
\makeatother
\end{itshape}

\pagenumbering{Roman}

\addcontentsline{toc}{section}{Vorwort} 
 % optional
\addcontentsline{toc}{section}{Danksagung} % optional

%Table of content
\renewcommand{\cftsecleader}{\cftdotfill{\cftdotsep}}

%\begin{onehalfspacing}
\cleardoublepage
\renewcommand{\contentsname}{Inhaltsverzeichnis}
\tableofcontents
\cleardoublepage
%\end{onehalfspacing}


\printglossary[type=\acronymtype,style=long3col, title=Abkürzungsverzeichnis, toctitle=Abkürzungsverzeichnis, nonumberlist=true] 

% adding list of figures
\cleardoublepage
\renewcommand{\listfigurename}{Abbildungsverzeichnis}
\addcontentsline{toc}{section}{\listfigurename} 
\listoffigures


% adding list of tables
\cleardoublepage
\renewcommand{\listtablename}{Tabellenverzeichnis}
\addcontentsline{toc}{section}{\listtablename}
\listoftables

\addcontentsline{toc}{section}{Zusammenfassung} 
\section*{Zusammenfassung}
\subsection*{Einleitung}
\lipsum[1]
\subsection*{Methoden}
\lipsum[1]
\subsection*{Ergebnisse}
\lipsum[1]
\subsection*{Diskussion}
\lipsum[1]
\addcontentsline{toc}{section}{Abstract}
\section*{Abstract}

\lipsum[1]
\addcontentsline{toc}{section}{Veröffentlichungen} 
\section*{Veröffentlichungen}

$\bullet$ Author 1 Name, Author 2 Name, and Author 3 Name, {``Article Title"}, \emph{Submitted to Conference/Journal Name}, further attributes.\\
$\bullet$ Author 1 Name, Author 2 Name, and Author 3 Name, {``Article Title"}, \emph{Submitted to Conference/Journal Name}, Mon. Year. \\
$\bullet$ Author 1 Name, Author 2 Name, and Author 3 Name, {``Article Title"}, \emph{Submitted to Conference/Journal Name}, Mon. Year.



% $$$$$$$$$$$$$$$$$  include your separate chapters   $$$$$$$$$$$$$$
\cleardoublepage
\pagenumbering{arabic}
\section{Einleitung}
\label{einleitung}

Hinführung zum Thema, Aufzeigen der Kenntnis-/Forschungslücke Begründung der Fragestellung, Zielsetzung und
Einschränkungen/Abgrenzungen

Abkürzungen werden wie folgt benutzt: \acrlong{mug}, welche mit \acrshort{mug} abekürzt wird.

\clearpage


\lipsum[1]

\subsection{Background}
\lipsum[1] \cite{key1}

\subsubsection{Subbackground}
\section{Methoden}
\label{methoden}

\lipsum[1]
\section{Ergebnisse}
\label{ergebnisse}

\lipsum[1]

\section{Diskussion}
\label{diskussion}

Antworten auf die Forschungsfragen, Vergleichende Erläuterungen, 
Schlussfolgerungen, 
kritische Reflexion/Einschränkungen zu Inhalt und Methode, Implikationen für Theorie und Praxis, Ausblick und Anregungen für weiterführende Arbeiten

\lipsum[1]


% $$$$$$$$$$$$$$$$$  Reference style  Starts $$$$$$$$$$$$$$
\bibliographystyle{vancouver}
\renewcommand\refname{Literaturverzeichnis}
\addcontentsline{toc}{section}{Literaturverzeichnis}
\bibliography{References}
% $$$$$$$$$$$$$$$$$  Reference style Ends  $$$$$$$$$$$$$$


% $$$$$$$$$$$$$$$$$  include Appendices   $$$$$$$$$$$$$$
\appendix
\addcontentsline{toc}{section}{Appendix}
\section*{Appendix}

Your content goes here.

% $$$$$$$$$$$$$$$$$  Appendix ends  $$$$$$$$$$$$$$

\end{document}
